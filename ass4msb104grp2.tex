% Options for packages loaded elsewhere
% Options for packages loaded elsewhere
\PassOptionsToPackage{unicode}{hyperref}
\PassOptionsToPackage{hyphens}{url}
\PassOptionsToPackage{dvipsnames,svgnames,x11names}{xcolor}
%
\documentclass[
  british,
  a4paper,
]{article}
\usepackage{xcolor}
\usepackage{amsmath,amssymb}
\setcounter{secnumdepth}{5}
\usepackage{iftex}
\ifPDFTeX
  \usepackage[T1]{fontenc}
  \usepackage[utf8]{inputenc}
  \usepackage{textcomp} % provide euro and other symbols
\else % if luatex or xetex
  \usepackage{unicode-math} % this also loads fontspec
  \defaultfontfeatures{Scale=MatchLowercase}
  \defaultfontfeatures[\rmfamily]{Ligatures=TeX,Scale=1}
\fi
\usepackage{lmodern}
\ifPDFTeX\else
  % xetex/luatex font selection
\fi
% Use upquote if available, for straight quotes in verbatim environments
\IfFileExists{upquote.sty}{\usepackage{upquote}}{}
\IfFileExists{microtype.sty}{% use microtype if available
  \usepackage[]{microtype}
  \UseMicrotypeSet[protrusion]{basicmath} % disable protrusion for tt fonts
}{}
\makeatletter
\@ifundefined{KOMAClassName}{% if non-KOMA class
  \IfFileExists{parskip.sty}{%
    \usepackage{parskip}
  }{% else
    \setlength{\parindent}{0pt}
    \setlength{\parskip}{6pt plus 2pt minus 1pt}}
}{% if KOMA class
  \KOMAoptions{parskip=half}}
\makeatother
% Make \paragraph and \subparagraph free-standing
\makeatletter
\ifx\paragraph\undefined\else
  \let\oldparagraph\paragraph
  \renewcommand{\paragraph}{
    \@ifstar
      \xxxParagraphStar
      \xxxParagraphNoStar
  }
  \newcommand{\xxxParagraphStar}[1]{\oldparagraph*{#1}\mbox{}}
  \newcommand{\xxxParagraphNoStar}[1]{\oldparagraph{#1}\mbox{}}
\fi
\ifx\subparagraph\undefined\else
  \let\oldsubparagraph\subparagraph
  \renewcommand{\subparagraph}{
    \@ifstar
      \xxxSubParagraphStar
      \xxxSubParagraphNoStar
  }
  \newcommand{\xxxSubParagraphStar}[1]{\oldsubparagraph*{#1}\mbox{}}
  \newcommand{\xxxSubParagraphNoStar}[1]{\oldsubparagraph{#1}\mbox{}}
\fi
\makeatother


\usepackage{longtable,booktabs,array}
\usepackage{calc} % for calculating minipage widths
% Correct order of tables after \paragraph or \subparagraph
\usepackage{etoolbox}
\makeatletter
\patchcmd\longtable{\par}{\if@noskipsec\mbox{}\fi\par}{}{}
\makeatother
% Allow footnotes in longtable head/foot
\IfFileExists{footnotehyper.sty}{\usepackage{footnotehyper}}{\usepackage{footnote}}
\makesavenoteenv{longtable}
\usepackage{graphicx}
\makeatletter
\newsavebox\pandoc@box
\newcommand*\pandocbounded[1]{% scales image to fit in text height/width
  \sbox\pandoc@box{#1}%
  \Gscale@div\@tempa{\textheight}{\dimexpr\ht\pandoc@box+\dp\pandoc@box\relax}%
  \Gscale@div\@tempb{\linewidth}{\wd\pandoc@box}%
  \ifdim\@tempb\p@<\@tempa\p@\let\@tempa\@tempb\fi% select the smaller of both
  \ifdim\@tempa\p@<\p@\scalebox{\@tempa}{\usebox\pandoc@box}%
  \else\usebox{\pandoc@box}%
  \fi%
}
% Set default figure placement to htbp
\def\fps@figure{htbp}
\makeatother


% definitions for citeproc citations
\NewDocumentCommand\citeproctext{}{}
\NewDocumentCommand\citeproc{mm}{%
  \begingroup\def\citeproctext{#2}\cite{#1}\endgroup}
\makeatletter
 % allow citations to break across lines
 \let\@cite@ofmt\@firstofone
 % avoid brackets around text for \cite:
 \def\@biblabel#1{}
 \def\@cite#1#2{{#1\if@tempswa , #2\fi}}
\makeatother
\newlength{\cslhangindent}
\setlength{\cslhangindent}{1.5em}
\newlength{\csllabelwidth}
\setlength{\csllabelwidth}{3em}
\newenvironment{CSLReferences}[2] % #1 hanging-indent, #2 entry-spacing
 {\begin{list}{}{%
  \setlength{\itemindent}{0pt}
  \setlength{\leftmargin}{0pt}
  \setlength{\parsep}{0pt}
  % turn on hanging indent if param 1 is 1
  \ifodd #1
   \setlength{\leftmargin}{\cslhangindent}
   \setlength{\itemindent}{-1\cslhangindent}
  \fi
  % set entry spacing
  \setlength{\itemsep}{#2\baselineskip}}}
 {\end{list}}
\usepackage{calc}
\newcommand{\CSLBlock}[1]{\hfill\break\parbox[t]{\linewidth}{\strut\ignorespaces#1\strut}}
\newcommand{\CSLLeftMargin}[1]{\parbox[t]{\csllabelwidth}{\strut#1\strut}}
\newcommand{\CSLRightInline}[1]{\parbox[t]{\linewidth - \csllabelwidth}{\strut#1\strut}}
\newcommand{\CSLIndent}[1]{\hspace{\cslhangindent}#1}

\ifLuaTeX
\usepackage[bidi=basic]{babel}
\else
\usepackage[bidi=default]{babel}
\fi
% get rid of language-specific shorthands (see #6817):
\let\LanguageShortHands\languageshorthands
\def\languageshorthands#1{}
\ifLuaTeX
  \usepackage[english]{selnolig} % disable illegal ligatures
\fi


\setlength{\emergencystretch}{3em} % prevent overfull lines

\providecommand{\tightlist}{%
  \setlength{\itemsep}{0pt}\setlength{\parskip}{0pt}}



 


\makeatletter
\@ifpackageloaded{caption}{}{\usepackage{caption}}
\AtBeginDocument{%
\ifdefined\contentsname
  \renewcommand*\contentsname{Table of contents}
\else
  \newcommand\contentsname{Table of contents}
\fi
\ifdefined\listfigurename
  \renewcommand*\listfigurename{List of Figures}
\else
  \newcommand\listfigurename{List of Figures}
\fi
\ifdefined\listtablename
  \renewcommand*\listtablename{List of Tables}
\else
  \newcommand\listtablename{List of Tables}
\fi
\ifdefined\figurename
  \renewcommand*\figurename{Figure}
\else
  \newcommand\figurename{Figure}
\fi
\ifdefined\tablename
  \renewcommand*\tablename{Table}
\else
  \newcommand\tablename{Table}
\fi
}
\@ifpackageloaded{float}{}{\usepackage{float}}
\floatstyle{ruled}
\@ifundefined{c@chapter}{\newfloat{codelisting}{h}{lop}}{\newfloat{codelisting}{h}{lop}[chapter]}
\floatname{codelisting}{Listing}
\newcommand*\listoflistings{\listof{codelisting}{List of Listings}}
\makeatother
\makeatletter
\makeatother
\makeatletter
\@ifpackageloaded{caption}{}{\usepackage{caption}}
\@ifpackageloaded{subcaption}{}{\usepackage{subcaption}}
\makeatother
\usepackage{bookmark}
\IfFileExists{xurl.sty}{\usepackage{xurl}}{} % add URL line breaks if available
\urlstyle{same}
\hypersetup{
  pdftitle={Assignment 4:},
  pdfauthor={Kristoffer Tufta and Harald Bjarne Vika},
  pdflang={en-GB},
  colorlinks=true,
  linkcolor={blue},
  filecolor={Maroon},
  citecolor={Blue},
  urlcolor={Blue},
  pdfcreator={LaTeX via pandoc}}


\title{Assignment 4:}
\usepackage{etoolbox}
\makeatletter
\providecommand{\subtitle}[1]{% add subtitle to \maketitle
  \apptocmd{\@title}{\par {\large #1 \par}}{}{}
}
\makeatother
\subtitle{Regional GDP Inequality in 4 Selected European Economies -
Synthesised Results.}
\author{Kristoffer Tufta and Harald Bjarne Vika}
\date{Friday 12 Dec, 2025}
\begin{document}
\maketitle


\section{Assignment 4 Consolidation of Key
findings}\label{assignment-4-consolidation-of-key-findings}

\subsection{Abstract}\label{abstract}

This paper looks at what drives internal economic disparities within
European NUTS2 regions. Using Eurostat data for Germany, Ireland,
Croatia, and Switzerland from 2000--2023, we measure how regional GDP
per capita growth relates to the spread of GDP across NUTS3-regions
inside each NUTS2-region. The analysis combines cross-sectional models,
structural segmentation (population density, workforce size,
unemployment), alternative functional forms, and fixed-effects panel
estimation.

The results point in one direction: short-term regional growth does not
meaningfully reduce internal GDP inequality. Cross-sectional regressions
show weak and unstable effects, and segmentation reveals that structural
characteristics matter far more than growth itself. High-density regions
are generally more balanced internally, while rural and low-density
regions show larger internal gaps. Workforce size and unemployment
conditions influence inequality patterns, but their effects vary across
models.

The panel estimates settle the picture. Once we control for
region-specific characteristics and common yearly shocks, the
growth--inequality relationship remains small. Internal disparities are
largely shaped by long-term structural factors, not short-term
performance.

Overall, the findings suggest that regional GDP inequality is a
structural issue. Broad growth policies alone are insufficient; more
targeted, place-based strategies are needed if the goal is to create
more balanced development inside regions.

\subsection{Introduction}\label{introduction}

\subsubsection{Background}\label{background}

Regional development in Europe is uneven, not just between the different
national economies but within the regions themselves. Many NUTS2 regions
contain both strong and weak local economies, and the internal spread of
GDP can be wide. These internal differences influence local labour
markets, municipal finances, service provision, and long-term
opportunities for residents. At the same time, regional inequality is
often discussed at the national level, even though the more practical
challenges, uneven access to jobs, productivity gaps, and infrastructure
differences, play out inside regions. Understanding what drives internal
regional GDP disparities is therefore important for anyone trying to
make regional development more balanced.

\subsubsection{Objectives}\label{objectives}

Assignments 1, 2, and 3 focused on different parts of this problem.

\begin{itemize}
\tightlist
\item
  \textbf{Assignment 1} built the dataset: collecting GDP, population
  and information at the NUTS2 level, and calculated the intra-regional
  GDP inequality using a population-weighted Gini coefficient.
\item
  \textbf{Assignment 2} analysed a cross-section of the year 2017 to see
  whether regional development, measured as GDP per capita growth, is
  linked to internal GDP inequality. Then find suitable variables for
  the same purpose.
\item
  \textbf{Assignment 3}looked at whether this relationship differs
  across different segments of population density, workforce size, and
  unemployment rates. We also extended the analysis to a full panel,
  estimating fixed-effects models to test whether the
  development--inequality relationship holds once we control for
  time-invariant regional characteristics and common yearly shocks. We
  also evaluated alternative functional forms and ran diagnostic tests
  to check model robustness.
\end{itemize}

Across all three assignments, the overarching research question has been
the same: \textbf{\emph{Does economic development lead to more balanced
regional GDP distribution?}}

\subsubsection{Significance}\label{significance}

This topic is relevant because internal regional disparities affect
everyday economic conditions more directly than national averages. When
only a few municipalities drive most of a region's GDP, the surrounding
areas may face weaker labour markets, lower tax bases, and fewer
opportunities due to the ``economic field of gravity'' of the dominating
region centralizing economic activity towards itself, even if the region
as a whole looks like it performs well.\\
Policymakers may then assume that economic growth alone may ``spill
over'' (or ``trickle down'', if you will) and reduce these gaps, but
this may not always be the case.

Our work tests that assumption with comparable data across four European
countries. By combining cross-sectional, segmented, and fixed-effects
panel approaches, the study provides a clearer view of whether
short-term development actually improves internal regional balance, or
whether structural characteristics dominate the pattern. This helps
clarify where growth policies are sufficient, and where more targeted,
place-based strategies are needed.

\subsection{Literature Review}\label{literature-review}

\subsubsection{Previous Work}\label{previous-work}

A central reference point for our work is Lessmann \& Seidel (2017), who
shows that the link between economic development and regional inequality
is not linear. Their results suggest something closer to an inverted-U
(and in some cases N-shaped) pattern, where inequality tends to rise in
the middle stages of development before falling again in high-income
settings. This gives a natural expectation that fast-growing regions may
show lower inequality if they are already relatively advanced, while
poorer or developing regions might move in the opposite direction. Their
work is also relevant because they rely on both traditional data and
satellite-based luminosity data, which shows that inequality can be
meaningfully captured even when official data is patchy --- a recurring
issue even in Eurostat datasets, as we experienced directly.

Eurostat's own methodological documentation is also part of the relevant
background. Their notes on missing values help explain why regional GDP
and population numbers are incomplete in certain years. These gaps arise
from reclassification of regions, confidentiality rules, or delayed
reporting. Understanding why these ``quirks'' happen is important
because it directly affects our results.

For our segmentation work, we also build on Eurostat's definitions of
population-density classes used for identifying rural areas, urban
clusters, and high-density urban centres. While Eurostat's approach
operates at the 1 km² continuous grid-cell level, their breakpoints (300
and 1500 people per square kilometer) give a practical structure for
grouping NUTS2 regions by density. This matters because urban density is
often correlated with both productivity and access to public-services,
and therefore a likely influence when it comes to regional GDP
inequality.

\subsubsection{Research Gap}\label{research-gap}

While there is existing literature on national GDP inequality trends and
on broader development patterns, there is less focus on how short-term
growth shocks play out inside individual NUTS2 regions, and even less on
whether the effect of growth differs across structural categories like
urban density, unemployment rates, or workforce size. Our assignments
aim to examine exactly that: to see whether any of these structural
categories influence GDP inequality in NUTS2-regions, and if so, to what
degree.

\subsection{Data and Methodology}\label{data-and-methodology}

Before we present our analyse and values from our data, we must first
see where our source data came from, their codes and any explanation on
why some of the datasets a missing data. Then we will go through the
approach on our assignments and how we analysed the data we collected
and transformed.

\subsubsection{Data Sources}\label{data-sources}

Primary data refers to information collected firsthand by the researcher
directly from original source. It is raw and unprocessed, typically
gathered through methods such as surveys, interviews and observations.
The data provided by Eurostat can be considered primary data because it
is originally collected through official surveys and processed following
established methodological standards for gathering information on
specific topic within defined time periods.

In the first assignment, we used Eurostat to extract datasets on gross
domestic product (GDP) at current market prices by NUTS 3 regions, with
the datacode: nama\_10r\_3gdp and population on 1 January by broad age
group, sex and NUTS 3 region, with the datacode: demo\_r\_pjanaggr3.
From these datasets, we collected data for Germany, Ireland, Croatia and
Switzerland covering the years 2000 to 2023.

In the second assignment, we retrieved data from three additional
datasets that could potentially influence regional inequality in 2017.
These datasets included labour force by NUTS 2 region with the datacode:
lfst\_r\_lfp2act, unemployment rate by Nuts 2 region with the data code:
tgs00010, and population density by NUTS 2 region with the data code:
tgs00024 . Each country collects its own data and reports it to Eurostat
in accordance with ESA 2010, a global framework for national accounting.
This ensures that the resulting statistics are consistent, reliable, and
comparable across countries (\emph{Information on Data - National
Accounts - Eurostat}, n.d.). The dataset labour force consists of all
available workforce in the region, and is a sum of both employed and
unemployed individuals, in thousand persons. We referes to this as both
workforce and labour force in the assignments.

However, the data also comes with certain limitations. For example,
Switzerland is not en EU member but an EFTA country (European Free Trade
Association), which promotes free trade and economic cooperation.
Because Switzerland did not have a formal datasharing agreement with
Eurostat before 2008, some data for earlier years or specific topics may
be missing (\emph{Information on Data - National Accounts - Eurostat},
n.d.).

Another limitation relates to confidentiality. Ireland, for instance,
lacks GDP data for 2015-2017 in the Mid-West and South-West regions due
to confidentiality restrictions imposed by Ireland's Central Statistics
Office. These restrictions are linked to changes in the national
accounts in 2015 that significantly affected the measurement of
Ireland's productive capacity (\emph{County Incomes and Regional GDP
2015 - CSO - Central Statistics Office}, 2018).

Some countries also choose to delay the publication of certain
statistics to ensure accuracy. As a result, data for the most recent
years of 2022 and 2023, may still be incomplete due to its
unavailability.

A common limitation across several datasets concerns changes in regional
boundaries. Regions may merge to form larger areas or split into smaller
units, affecting how data is collected, aggregated and updated. This
issue is particularly relevant in Germany, where regional reforms occur
more frequently due to regional policy. Croatia and Ireland face similar
challenges, as changes to their NUTS regions have affected population
data availability at the NUTS 3 level ({`The NUTS Classification in
Croatia'}, n.d.).

Finally, there were limitations related to the time coverage of the
datasets. Allthought the assignments aimed to examine data from 2000 to
2023, the population density dataset for NUTS 2 regions only covers
2012-2023, and the unemployment rate dataset covers 2013-2023.
Consequently, when including these variables, we were restricted to
using data from 2013 to 2023 to maintain consistency across datasets.

\subsubsection{Methodological approach}\label{methodological-approach}

\paragraph{Cross-Sectional Estimation}\label{cross-sectional-estimation}

We first combined the Eurostat datasets covering GDP (in million euros)
and population, , then calculated GDP per capita and Gini coefficients
using a formula weighted by population, like Lessmann \& Seidel (2017).
We then visualized the calculated Gini coefficients for the different
regions using different plots from ggplot2. In assignment 2, we made a
cross-sectional analysis of the year 2017 using a regression model to
test the effect of change in regional GDP per capita on regional
inequality. We specified the following linear regression model: \[
\text{Gini}_{i,2017} = \alpha + \beta \, \Delta \text{GDPpc}_{i,2016 \to 2017} + u_i
\] Where:

\begin{itemize}
\tightlist
\item
  \(\text{Gini}_{i,2017}\) --- Calculated Gini coefficient of region
  \emph{i} in 2017 (regional GDP per capita inequality)\\
\item
  \(\Delta \text{GDPpc}_{i,2016 \to 2017}\) --- percent change in GDP
  per capita from 2016 to 2017 in region \emph{i}\\
\item
  \(\alpha\) --- intercept term (baseline inequality when GDP-per-capita
  change is zero)\\
\item
  \(\beta\) --- slope coefficient showing how inequality changes with a
  one-percentage-point increase in GDP-per-capita growth\\
\item
  \(u_i\) --- error term capturing unobserved regional factors
\end{itemize}

We then experimented with other determinants of inequality by making a
new OLS model with a set of new variables: population density (persons
per square kilometer), unemployment rate and total workforce, also
fetched from Eurostat. We specified the MLR as follows: \[
\text{Gini}_{i,2017} = \alpha + \beta_1\,\text{Workforce}_i + \beta_2\,\text{PopDensity}_i + \beta_3\,\text{Unemployment}_i + u_i\]

where \(\alpha\) is the intercept, \(\beta_1\), \(\beta_2\), and
\(\beta_3\) are the slope coefficients for each explanatory variable,
and \(u_i\) is the error term capturing unobserved factors affecting
inequality.

\paragraph{Segmentation}\label{segmentation}

We also segmented regions into categories to examine if the
development--inequality relationship differed across different
structural environments. Population density was split into ``Rural'',
``Medium Density'', and ``High Density'' using Eurostat-inspired
breakpoints on population per km² (0--300, 300--1500, 1500+). For
workforce size, we used quantile-based grouping (Low, Medium, High) that
split our data into three equally populated groups. Unemployment rate
was also split into Low-High segments, but here we set the breakpoints
manually at 0\%-5\%, 5\%-10\% and 10+\%. These subsets were used to run
separate regressions for each group.

\paragraph{Panel Estimation}\label{panel-estimation}

Finally, we estimated fixed-effects models using the full panel,
applying plm() with region, country, year, and two-way effects. Before
estimation, we converted the dataset into a pdata.frame and cleaned
remaining NA/Inf values to avoid dropped regions. The panel models
allowed us to control for all time-invariant regional characteristics,
giving a ``less noisy'' picture of how regional GDP inequality changes
inside regions over time.

\paragraph{Handling of NA-values and
Heteroskedasticity}\label{handling-of-na-values-and-heteroskedasticity}

After joining the datasets onto eachother, including the new variables
used for segmenting, we ended up with a few obvious gaps in our data and
some NA/Inf rows. These were removed to keep the panel consistent, but a
side effect of this is that we only have panel data from 2013 onwards.

We also ran Breusch--Pagan tests to check for heteroskedasticity across
all models and found no statistically significant signs of
heteroskedasticity.\\
Where alternative functional forms were required, we introduced
quadratic and logarithmic transformations of the key variables and
evaluated their residual behavior the same way.

\subsection{Empirical Findings}\label{empirical-findings}

\subsubsection{Cross- sectional
Estimates}\label{cross--sectional-estimates}

In assignment 2, we began with a cross-sectional analysis examining the
relationship between regional inequality and short-term economic
development, measured as the change in GDP per capita from 2016 to 2017.
The dependent variable in our model was regional inequality, measured by
the Gini coefficient, and the independent variable was the change in GDP
per capita: \[
\text{Gini}_{i,2017} = \alpha + \beta \, \Delta \text{GDPpc}_{i,2016 \to 2017} + u_i\]

\begin{figure}[H]

{\centering \pandocbounded{\includegraphics[keepaspectratio]{ass2msb104grp2_files/figure-html/fig-RegDev-Gini-1.png}}

}

\caption{OLS Estimate of Regional Inequality as a function of GDP per
Capita Growth (\%).}

\end{figure}%

From the linear regression model we get the following results:

\begin{itemize}
\tightlist
\item
  When change gdp per capita growth is zero, the regional inequality
  coefficient is estimated \(\hat{\alpha}=0.134\).
\item
  Change in in GDP per capita growth has a coefficient of
  \(\hat{\beta}=-0.002\) (p=0.435), indicating it is not statistically
  significant.
\item
  Overall, the model's explanatory power is very low: \(R^2 = 0.012\),
  adjusted \(R^2 = -0.008\), and the F-test (F = 0.619, p = 0.435)
  suggests the model is not statistically significant.
\end{itemize}

Thus, although the coefficient suggests a negative relationship, the
effect is statistically insignificant and not substantively meaningful.

We then went to test the three other variables, labour force,
unemployment rate and population density, to determine whether they
serve as strong predictor of regional inequality then GDP-per-capita
growth. After cleaning and tidying the data, we created a multiple
regression model to analyse their cross-sectional relationship with
inequality: \[
\text{Gini}_{i,2017} = \alpha + \beta_1\,\text{Workforce}_i + \beta_2\,\text{PopDensity}_i + \beta_3\,\text{Unemployment}_i + u_i\]

Running the regression model gave us the following estimates:

\begin{itemize}
\tightlist
\item
  When all independent variables are zero, the regional inequality
  coefficient is estimated at \(\beta = 0.150\) (p \textless{} 0.001).
\item
  Workforce has a coefficient of \(\beta = 0.000043\) (p = 0.0216),
  statistically significant at the 5\% level, indicating that regional
  inequality increases slightly as the labour force grows by one
  thousand workers.
\item
  Population density (Pop\_km²) has a coefficient of
  \(\beta = -0.000041\) (p = 0.0083), statistically significant at the
  1\% level, suggesting that more densely populated regions tend to have
  lower inequality.
\item
  Unemployment (Unemp\_prct) has a coefficient of \(\beta = -0.0114\) (p
  = 0.0613), marginally significant at the 10\% level, indicating a weak
  negative relationship with inequality.
\item
  Overall, the model's explanatory power is moderate: \(R^2 = 0.283\),
  adjusted \(R^2 = 0.235\), and the F-test (F = 5.908, p = 0.0017)
  indicates the model is statistically significant. Structural variables
  such as workforce size and population density better capture regional
  economic differences than short-term GDP-per-capita growth.
\end{itemize}

\subsubsection{Alternative functional forms and panel
estimates}\label{alternative-functional-forms-and-panel-estimates}

In this section, we explored alternative functional forms of the
regression models using the variables from the second assignment
workforce, population density, and unemployment rate to examine their
effects on regional inequality, measured by the Gini coefficient.
Specifically, we considered linear, logarithmic, and quadratic
functional forms. While cubic specifications were also tested, they
produced more estimates without providing meaningful improvements in
explanatory power, so they were not pursued further.

We constructed separate regression tables for each functional form
linear, logarithmic, quadratic and compared their performance in terms
of coefficient significance, explanatory power, and overall model fit.
Below, we summarize the key findings and interpret the effects of each
independent variable.

\paragraph{Labour force}\label{labour-force}

Linear model:

\begin{itemize}
\tightlist
\item
  When all independent variables are zero, the regional inequality
  coefficient is estimated at \(\beta = 0.1773\).
\item
  Workforce has a coefficient of \(\beta = 0.000036\) (p = 0.0665),
  which is marginally significant at the 10\% level.
\item
  Change in GDP per capita has a coefficient of \(\beta = -0.002122\) (p
  = 0.4931), indicating it is not statistically significant.
\item
  Overall, the model's explanatory power is low: \(R^2 = 0.074\),
  adjusted \(R^2 = 0.035\), and the F-test (F = 1.877, p = 0.163)
  suggests the model is not statistically significant.
\end{itemize}

Logarithmic model:

\begin{itemize}
\tightlist
\item
  Transforming change in GDP per capita using a log transformation
  improves model performance.
\item
  When all independent variables are zero, the regional inequality
  coefficient is estimated at \(\beta = 0.0.2292\).
\item
  The coefficient for GDP per capita becomes \(\beta = -0.059611\) and
  is highly significant (p = 0.0008), showing a negative relationship
  with the Gini coefficient.
\item
  Workforce remains largely unchanged at \(\beta = 0.000033\) (p =
  0.0604, marginally significant).
\item
  Explanatory power improves: \(R^2 = 0.288\), adjusted
  \(R^2 = 0.0251\), and the F-test is highly significant (F = 8.078, p =
  0.0011).
\end{itemize}

Quadratic model:

\begin{itemize}
\tightlist
\item
  The quadratic specification produces less significant results than the
  logarithmic model.
\item
  When all independent variables are zero, the regional inequality
  coefficient is estimated at \(\beta = 0.1379\).
\item
  GDP per capita: linear term \(\beta = 0.004327\) (p = 0.4066),
  quadratic term \(\beta = -0.000468\) (p = 0.1302), both insignificant.
\item
  Workforce: \(\beta = 0.000029\) (p = 0.1480), effect is weak.
\item
  \(R^2 = 0.1194\), adjusted \(R^2 = 0.06199\), and the model is not
  statistically significant overall (p = 0.116).
\end{itemize}

Summary: The logarithmic model best captures the relationship between
workforce, GDP growth, and inequality, mainly due to the strong
significance of log-transformed GDP per capita.

\paragraph{Population Density}\label{population-density}

Linear model:

\begin{itemize}
\tightlist
\item
  When change in GDP per capita and population density are zero, the
  regional inequality coefficient is estimated at \(\beta = 0.1506\).
\item
  Population density has a coefficient of \(\beta = -0.000037\) with a
  p-value of 0.0228, indicating statistical significance at the 5\%
  level.
\item
  Change in GDP per capita has a coefficient of \(\beta = 0.000054\) (p
  = 0.6166), which is not significant.
\item
  Model performance is limited: \(R^2 = 0.111\), adjusted
  \(R^2 = 0.072\), and the F-test is 7.36 with a p-value of 0.00668,
  suggesting marginal overall significance.
\end{itemize}

Logarithmic model:

\begin{itemize}
\tightlist
\item
  Applying a log transformation to change in GDP per capita improves
  model fit and significance.
\item
  When change in GDP per capita and population density are zero, the
  regional inequality coefficient is estimated at \(\beta = 0.229228\).
\item
  GDP per capita becomes \(\beta = -0.05793\), p = 0.0007, showing a
  strong negative relationship with inequality.
\item
  Population density remains similar: \(\beta = -0.000039\), p = 0.0047,
  still significant.
\item
  Overall explanatory power increases substantially: \(R^2 = 0.364\),
  adjusted \(R^2 = 0.331\), and the F-test p-value is 0.0001, indicating
  a highly significant model.
\end{itemize}

Quadratic model:

\begin{itemize}
\tightlist
\item
  Including a quadratic term for GDP per capita allows for a non-linear
  relationship with inequality.
\item
  When change in GDP per capita and population density are zero, the
  regional inequality coefficient is estimated at \(\beta = 0.137873\).
\item
  GDP per capita: linear term \(\beta = 0.0077\) (p = 0.1136, not
  significant), quadratic term \(\beta = -0.000682\) (p = 0.0198,
  significant), indicating a slight non-linear effect.
\item
  Population density: \(\beta = -0.000041\), p = 0.0091, remains
  significant.
\item
  Model fit is moderate: \(R^2 = 0.213\), adjusted \(R^2 = 0.161\), with
  overall model p = 0.0123, showing statistical significance.
\end{itemize}

Summary: The logarithmic model provides the best explanatory power for
population density, with both GDP growth and population density showing
significant impacts on regional inequality. The quadratic model captures
a slight non-linear effect but has lower overall explanatory power.

\paragraph{Unemployment Rate}\label{unemployment-rate}

Linear model:

\begin{itemize}
\tightlist
\item
  The regional inequality coefficient is \(\beta = 0.1758\) when GDP per
  capita growth and unemployment are zero.
\item
  Unemployment rate: \(\beta = -0.001039\), p = 0.03, statistically
  significant, suggesting higher unemployment is associated with
  slightly lower inequality.
\item
  Change in GDP per capita: \(\beta = -0.0000248\), p = 0.936, not
  significant.
\item
  Model fit is weak: \(R^2 = 0.10\), adjusted \(R^2 = 0.062\), and the
  F-test p-value is 0.083, indicating borderline overall significance.
\end{itemize}

Logarithmic model:

\begin{itemize}
\tightlist
\item
  Transforming GDP per capita using a log improves model performance.
\item
  The regional inequality coefficient is \(\beta = 0.233016\) when GDP
  per capita growth and unemployment are zero.
\item
  GDP per capita: \(\beta = -0.051\), p = 0.005, now highly significant
  with a negative effect on inequality.
\item
  Unemployment rate: \(\beta = -0.00707\), p = 0.1096, not significant,
  though the effect remains negative.
\item
  Explanatory power increases: \(R^2 = 0.27\), adjusted
  \(R^2 = 0.2336\), F-test p = 0.0018, highly significant overall.
\end{itemize}

Quadratic model:

\begin{itemize}
\tightlist
\item
  Including a quadratic term for GDP per capita allows for a potential
  non-linear effect.
\item
  The regional inequality coefficient is \(\beta = 0.160671\) when GDP
  per capita growth and unemployment are zero.
\item
  GDP per capita: linear term \(\beta = 0.0062\) (p = 0.2114, not
  significant), quadratic term \(\beta = -0.000488\) (p = 0.0102,
  borderline significant).
\item
  Unemployment rate: \(\beta = -0.009216\), p = 0.0516, borderline
  significant.
\item
  Model fit is lower than the log model: \(R^2 = 0.1517\), adjusted
  \(R^2 = 0.09633\), with F-test p = 0.0539, borderline significant
  overall.
\end{itemize}

Summary: The logarithmic transformation of GDP per capita produces the
strongest model for unemployment, improving both explanatory power and
overall significance. Unemployment shows a weak but negative
relationship with inequality, while quadratic effects of GDP per capita
are borderline significant.

Across all variables, logarithmic models generally provide the best
explanatory power and statistical significance, particularly through the
transformation of GDP per capita. Quadratic models capture some
non-linear effects but often at the cost of reduced overall
significance. Linear models are generally the weakest in explaining
regional inequality. The key takeaway is that GDP per capita growth
consistently shows a negative relationship with inequality when
appropriately transformed, while workforce, population density, and
unemployment rate exhibit more nuanced, context-dependent effects.

The panel regression results reveal a consistently positive association
between change in GDP per capita and the dependent variable across all
specifications. The magnitude of the effect is small (0.00017--0.00024),
but precision improves substantially when controlling for regional
heterogeneity using NUTS2 fixed effects, as reflected in markedly lower
standard errors and RMSE. Models with country only or year only fixed
effects explain very little variation (R² = 0.005--0.009), whereas NUTS2
and NUTS2 \& Year fixed effects substantially improve model fit (R² =
0.102--0.109) and are preferred according to AIC and BIC. These findings
highlight the importance of accounting for subnational regional
characteristics, which capture significant variation in the outcome
variable. Overall, the results suggest that regional-level factors are
key drivers of the dependent variable, while the positive effect of
change in GDP per capita remains robust across specifications.

\subsection{Discussion}\label{discussion}

\subsubsection{Key insights}\label{key-insights}

We will here take our key values from earlier in the project and discuss
their effect on regional inequality. The initial model examining the
effect of change in GDP per capita on regional inequality shows very
weak explanatory capacity. The model's F-test yields a p-value of 0.435,
indicating that the relationship is statistically insignificant, and the
multiple R² of 1.2\% demonstrates that GDP per capita growth accounts
for almost none of the variation in inequality. The estimated
coefficient is negative (β = --0.002), but its magnitude is negligible,
and it fails to reach significance. This suggests that short-term
economic growth on its own does not meaningfully explain differences in
regional inequality.

When additional variables such as labour force, population density, and
unemployment rate are included, the explanatory power improves
substantially. The multiple R² increases to 0.2826, with an adjusted R²
of 0.2357, indicating that these factors together account for 23--28\%
of the variation in regional inequality. Each predictor contributes
differently: workforce is significant at the 5\% level (p = 0.0216),
population density at the 1\% level (p = 0.0083), and unemployment is
marginally significant at the 10\% level (p = 0.0613). The coefficients
indicate a small positive association for workforce (β = 0.000043),
while population density (β = --0.000041) and unemployment (β =
--0.0114) gives a negative relationships with inequality. These patterns
suggest that regions with larger labour forces may experience slightly
higher inequality, whereas denser regions tend to show lower inequality,
and higher unemployment appears to compress income distribution. The
estimated value of regional inequality (β = 0.150168) cannot be
meaningfully interpreted, as no region can give a zero value for any of
the three predictors.

Using alternative functional form would give a new insight into how the
independent variables has a effect on the dependent variable of regional
inequality. In the workforce models, the linear specification estimates
the intercept at β = 0.1773, indicating the baseline level of inequality
when all predictors are zero. Workforce has a small positive coefficient
(β = 0.000036, p = 0.0665), marginally significant at the 10\% level.
This suggests that for every additional unit of workforce, such as an
extra thousand workers, the regional Gini coefficient increases by
0.000036, representing a very slight rise in inequality. GDP per capita
growth in the linear model is insignificant (β = --0.002122, p =
0.4931), and overall explanatory power is low (R² = 0.074, adjusted R² =
0.035; F = 1.877, p = 0.163).

The logarithmic specification improves the model substantially. The
intercept increases to β = 0.2292, and log-transformed GDP per capita
becomes highly significant (β = --0.059611, p = 0.0008), indicating that
proportional increases in GDP per capita are associated with reductions
in inequality. Workforce remains marginally significant (β = 0.000033, p
= 0.0604), implying that a one-unit increase still slightly raises
inequality, but the effect is very small relative to the impact of GDP
growth. Explanatory power rises to R² = 0.288, adjusted R² = 0.251, with
an F-test of 8.078 (p = 0.0011). The quadratic model reduces
significance: GDP linear (β = 0.004327, p = 0.4066) and quadratic terms
(β = --0.000468, p = 0.1302) are insignificant, workforce becomes weaker
(β = 0.000029, p = 0.1480), and overall model fit is low (R² = 0.1194,
adjusted R² = 0.06199, p = 0.116).

For population density, the linear model estimates the intercept at β =
0.1506, with population density negatively associated with inequality (β
= --0.000037, p = 0.0228). A one-unit increase in population density
(e.g., one person per square kilometer) decreases the Gini coefficient
by 0.000037, indicating slightly lower inequality in denser regions. GDP
per capita is insignificant (β = 0.000054, p = 0.6166), and model fit is
modest (R² = 0.111, adjusted R² = 0.072; F = 7.36, p = 0.00668). The
logarithmic model strengthens the relationships: intercept rises to β =
0.229228, log GDP per capita is highly significant and negative (β =
--0.05793, p = 0.0007), and population density remains negative and
significant (β = --0.000039, p = 0.0047). Explanatory power increases
substantially (R² = 0.364, adjusted R² = 0.331, F-test p = 0.0001). The
quadratic model introduces a significant squared GDP term (β =
--0.000682, p = 0.0198) while the linear term is not significant (β =
0.0077, p = 0.1136), and population density remains negative (β =
--0.000041, p = 0.0091). Model fit is moderate (R² = 0.213, adjusted R²
= 0.161, p = 0.0123), showing some curvature but weaker explanatory
power than the logarithmic specification.

In the unemployment models, the linear specification estimates the
intercept at β = 0.1758, with unemployment negatively associated with
inequality (β = --0.001039, p = 0.03). A one-percentage-point increase
in unemployment reduces the Gini coefficient by 0.001039, indicating a
slight compression of inequality. GDP per capita remains insignificant
(β = --0.0000248, p = 0.936), and model fit is low (R² = 0.10, adjusted
R² = 0.062, F-test p = 0.083). The logarithmic model improves
performance: intercept rises to β = 0.233016, log GDP per capita is
significant and negative (β = --0.051, p = 0.005), and unemployment
becomes insignificant (β = --0.00707, p = 0.1096). Model fit increases
(R² = 0.27, adjusted R² = 0.2336, F-test p = 0.0018). The quadratic
model shows borderline significance for unemployment (β = --0.009216, p
= 0.0516) and the squared GDP term (β = --0.000488, p = 0.0102), with
lower explanatory power (R² = 0.1517, adjusted R² = 0.09633, F-test p =
0.0539).

Overall, these results show that a one-unit increase in workforce
slightly raises inequality, higher population density slightly reduces
inequality, and higher unemployment slightly reduces inequality, while
log-transformed GDP per capita consistently reduces inequality across
the models. The magnitude of the GDP effect is much larger than the
other predictors, highlighting that proportional economic growth has a
stronger association with regional inequality than the small incremental
effects of workforce, population density, or unemployment.

The panel regression results add further nuance. Across specifications,
the estimated effect of change in GDP per capita is consistently
positive, with coefficients ranging from 0.00017 to 0.00024, though
these effects are small. The level of statistical precision improves
markedly when regional heterogeneity is controlled using NUTS2 fixed
effects. Models with only country or year fixed effects explain very
little variation (R² = 0.005--0.009), while those incorporating NUTS2
fixed effects achieve considerably higher explanatory power (R² =
0.102--0.109) and lower RMSE. The improvements in information criteria
(AIC and BIC) further indicate that regional characteristics play an
important role in shaping inequality outcomes.

Throughout these results, the differences in functional form,
particularly the impact of logarithmic versus linear specifications
demonstrate how sensitive the estimated relationships are to model
structure. Transformations applied to GDP per capita growth consistently
produce stronger, clearer relationships with inequality, while the
effects of workforce size, population density, and unemployment vary
considerably depending on the specification used.

\subsubsection{Policy Implications}\label{policy-implications}

Across all the cross-sectional and panel models, the effect of GDP per
capita growth on inequality is small, unstable, and mostly
insignificant. For policymakers, this means that betting on regional
inequality improvements through ``more growth'' alone would be a
sub-optimal approach. The differences between regions appear to be
driven far more by long-term structural characteristics than by
year-to-year economic performance.

The segmentation exercises support this interpretation, although with a
small asterisk attached. Regions with higher population density tend to
show slightly lower inequality, while rural regions display more
variation and, in some cases, higher inequality. This suggests that
sparsely populated areas face structural disadvantages---smaller labour
markets, fewer high-wage occupations, weaker public services---that
growth alone cannot compensate for. At the same time, the way
NUTS2-regions are defined (generally based on existing administrative
divisions), makes it so that we cannot take this at face value. This
division tends to separate city-regions into their own separate
NUTS2-regions, meaning that the most densely populated areas also by
definition have the least regional GDP inequality, seeing as they tend
to only consist of one region, and thus come out with a calculated Gini
of zero.

Regions with larger labour forces tend to show wider internal GDP
variation. This is consistent with regions that contain both highly
productive sub-areas and weaker ones. Policymakers in these regions may
need to invest specifically in connecting the smaller municipalities to
the dominant ``key region'' through commuting infrastructure, business
support, or coordinated urban planning.

The unemployment segmentation reinforces the point that ``low
disparities'' are not always a sign of strength. Regions with weaker
economic activity often show very low internal differences simply
because no municipality stands out as a strong performer. Policymakers
should avoid interpreting these cases as success stories and instead
view them as indicators of stagnation.

Finally, the panel results show that once we control for each region's
structural characteristics, the effect of growth on internal disparities
barely changes. The implication is: regional GDP inequality is mainly
driven by long-term spatial and economic structures, not by short-term
GDP growth. Effective policies therefore need to focus on durable
place-based strategies---strengthening urban centres, improving
connectivity, and reducing structural disadvantages---rather than
relying on general growth policies to equalise development inside
regions.

\subsection{Limitations and Future
Research}\label{limitations-and-future-research}

Through the analyse of the datasets from the three assignments, several
limitations emerged. These must be acknowledged when one shall
interpreting the results. The first limitation and the most crucial one
is the limited data across the different Eurostat datasets. Some
variables were not consistently reported for each country as mentioned.
These missing values gave us NA's in our coding and was challenging to
form a fully comparable and comprehensive dataset. This also crated
uncertainty regarding which variables were most suitable for capturing
the specific characteristics of the countries included in the study.

Another limitations is the knowledge of which variable would have the
greatest influence on regional inequalities, and how they will shape the
analysis. Without stronger evidence one variable impact, some
assumptions had to be made when selecting and interpreting the data.
Each countries followed a universe dictation on how to collect data for
Eurostat, but how the methods would be interpreted and executed across
each countries would further complicate comparisons, as each nation may
decipher the definitions, or reporting practices differently. These
inconsistencies can affect the reliability of cross-country analyses.

By gaining a deeper insight into the context of each country and how
they processes their collected data, we would have a stronger
interpretation of our results from the analyse. For a more accurate and
improved analysis of regional dynamics, a more in-depth understanding of
data practices and how these variables affects the countries would
likely improve the workflow and understanding of the results. This could
help with future analyses and enhance the ability to explain the
variations in regional inequalities.

\subsection{Conclusion}\label{conclusion}

\subsubsection{Summary}\label{summary}

Across all assignments, the core finding is straightforward: short-term
GDP per capita growth does not explain much of the internal GDP spread
within NUTS2 regions. The initial cross-sectional models showed almost
no relationship between growth and inequality. Once we introduced
workforce size, population density, and unemployment, the models
improved, but the effects were still small and varied across
specifications. Density consistently pointed toward slightly lower
inequality, while larger workforces tended to be linked with slightly
higher inequality. Unemployment effects were weak and context dependent.

The alternative functional forms reinforced this picture.
Log-transforming GDP per capita produced the clearest and most stable
relationships, but even then, the general message stayed the same:
proportional growth has some association with inequality, but the effect
is limited. Quadratic models captured small non-linearities but did not
change the overall interpretation.

The panel models helped settle the question. Once we control for
regional characteristics with fixed effects, most of the variation in
inequality is tied to long-lasting structural factors inside each
region, not year-to-year economic performance. Growth barely shifts the
needle. NUTS2 fixed effects improved the model fit dramatically, which
shows how important these structural differences are.

\subsubsection{Final Reflection}\label{final-reflection}

What this means is that internal regional inequality is not a simple
function of annual economic growth. Like Newton said: ``an object in
motion, stays in motion.'' The same goes here: the regions that start
out unequal tend to stay unequal, regardless of short-term swings in GDP
per capita. Regions with structural disadvantages, especially rural or
low-density areas, carry these disadvantages from year to year. Growth
helps, but it does not rebalance a region on its own. The implication
for future research is clear: to understand why some regions have wide
internal GDP spreads, we need richer variables that capture long-term
spatial, institutional, and historical differences, not just economic
snapshots.

\subsection{Appendix}\label{appendix}

In this assignment we have used AI to ask controlling questions and
constructive judging of the text and the codes used, including
interpreting the results from our models. The language models used were
ChatGPT 3.5 and ChatGPT 5.1. Our prompt strategy revolved around
describing the assignment to chatGPT, and asking it for feedback
underways. For model interpretation, we uploaded our rendered
html-documents, which included the results from our models, and asked
for assistance in interpretation. It was important for us to actually
learn, and not just outsource the thinking to an AI, so where ChatGPT
presented us with new concepts or R-functions, we asked it to educate us
and explain, step by step, what each new concept/function was and how it
could be used.

\subsection{Package Citations}\label{package-citations}

We used R v. 4.5.1 (R Core Team, 2025) and the following R packages:
dineq v. 0.1.0 (Schulenberg, 2018), flextable v. 0.9.10 (Gohel \&
Skintzos, 2025), lmtest v. 0.9.40 (Zeileis \& Hothorn, 2002),
modelsummary v. 2.5.0 (Arel-Bundock, 2022), officer v. 0.7.0 (Gohel et
al., 2025), plm v. 2.6.7 (Croissant \& Millo, 2008, 2018; Millo, 2017),
psych v. 2.5.6 (William Revelle, 2025), PxWebApiData v. 1.1.0 (Langsrud
\& Bruusgaard, 2025), rmarkdown v. 2.29 (Allaire et al., 2024; Xie et
al., 2018, 2020), tidyverse v. 2.0.0 (Wickham et al., 2019).

\subsection*{References}\label{references}
\addcontentsline{toc}{subsection}{References}

\phantomsection\label{refs}
\begin{CSLReferences}{1}{0}
\bibitem[\citeproctext]{ref-rmarkdown2024}
Allaire, J., Xie, Y., Dervieux, C., McPherson, J., Luraschi, J., Ushey,
K., Atkins, A., Wickham, H., Cheng, J., Chang, W., \& Iannone, R.
(2024). \emph{{rmarkdown}: Dynamic documents for r}.
\url{https://github.com/rstudio/rmarkdown}

\bibitem[\citeproctext]{ref-modelsummary}
Arel-Bundock, V. (2022). {modelsummary}: Data and model summaries in
{R}. \emph{Journal of Statistical Software}, \emph{103}(1), 1--23.
\url{https://doi.org/10.18637/jss.v103.i01}

\bibitem[\citeproctext]{ref-countyi2018}
\emph{County incomes and regional GDP 2015 - CSO - central statistics
office}. (2018). CSO.
\url{https://www.cso.ie/en/releasesandpublications/er/cirgdp/countyincomesandregionalgdp2015/}

\bibitem[\citeproctext]{ref-plm2008}
Croissant, Y., \& Millo, G. (2008). Panel data econometrics in {R}: The
{plm} package. \emph{Journal of Statistical Software}, \emph{27}(2),
1--43. \url{https://doi.org/10.18637/jss.v027.i02}

\bibitem[\citeproctext]{ref-plm2018}
Croissant, Y., \& Millo, G. (2018). \emph{Panel data econometrics with
{R}}. Wiley.

\bibitem[\citeproctext]{ref-officer}
Gohel, D., Moog, S., \& Heckmann, M. (2025). \emph{{officer}:
Manipulation of microsoft word and PowerPoint documents}.
\url{https://doi.org/10.32614/CRAN.package.officer}

\bibitem[\citeproctext]{ref-flextable}
Gohel, D., \& Skintzos, P. (2025). \emph{{flextable}: Functions for
tabular reporting}.
\url{https://doi.org/10.32614/CRAN.package.flextable}

\bibitem[\citeproctext]{ref-informat}
\emph{Information on data - national accounts - eurostat}. (n.d.).
\url{https://ec.europa.eu/eurostat/web/national-accounts/information-data}

\bibitem[\citeproctext]{ref-PxWebApiData}
Langsrud, Ø., \& Bruusgaard, J. (2025). \emph{{PxWebApiData}: PX-web
data by API}. \url{https://doi.org/10.32614/CRAN.package.PxWebApiData}

\bibitem[\citeproctext]{ref-lessmann2017}
Lessmann, C., \& Seidel, A. (2017). Regional inequality, convergence,
and its determinants -- a view from outer space. \emph{European Economic
Review}, \emph{92}, 110--132.
\url{https://doi.org/10.1016/j.euroecorev.2016.11.009}

\bibitem[\citeproctext]{ref-plm2017}
Millo, G. (2017). Robust standard error estimators for panel models: A
unifying approach. \emph{Journal of Statistical Software}, \emph{82}(3),
1--27. \url{https://doi.org/10.18637/jss.v082.i03}

\bibitem[\citeproctext]{ref-base}
R Core Team. (2025). \emph{{R}: A language and environment for
statistical computing}. R Foundation for Statistical Computing.
\url{https://www.R-project.org/}

\bibitem[\citeproctext]{ref-dineq}
Schulenberg, R. (2018). \emph{{dineq}: Decomposition of (income)
inequality}. \url{https://doi.org/10.32614/CRAN.package.dineq}

\bibitem[\citeproctext]{ref-thenuts}
The NUTS classification in croatia. (n.d.). In \emph{dzs.gov.hr}.
\url{https://dzs.gov.hr/highlighted-themes/prostorne-klasifikacije-i-subnacionalne-statistike-2-694/the-nuts-classification-in-croatia/699}

\bibitem[\citeproctext]{ref-tidyverse}
Wickham, H., Averick, M., Bryan, J., Chang, W., McGowan, L. D.,
François, R., Grolemund, G., Hayes, A., Henry, L., Hester, J., Kuhn, M.,
Pedersen, T. L., Miller, E., Bache, S. M., Müller, K., Ooms, J.,
Robinson, D., Seidel, D. P., Spinu, V., \ldots{} Yutani, H. (2019).
Welcome to the {tidyverse}. \emph{Journal of Open Source Software},
\emph{4}(43), 1686. \url{https://doi.org/10.21105/joss.01686}

\bibitem[\citeproctext]{ref-psych}
William Revelle. (2025). \emph{{psych}: Procedures for psychological,
psychometric, and personality research}. Northwestern University.
\url{https://CRAN.R-project.org/package=psych}

\bibitem[\citeproctext]{ref-rmarkdown2018}
Xie, Y., Allaire, J. J., \& Grolemund, G. (2018). \emph{R markdown: The
definitive guide}. Chapman; Hall/CRC.
\url{https://bookdown.org/yihui/rmarkdown}

\bibitem[\citeproctext]{ref-rmarkdown2020}
Xie, Y., Dervieux, C., \& Riederer, E. (2020). \emph{R markdown
cookbook}. Chapman; Hall/CRC.
\url{https://bookdown.org/yihui/rmarkdown-cookbook}

\bibitem[\citeproctext]{ref-lmtest}
Zeileis, A., \& Hothorn, T. (2002). Diagnostic checking in regression
relationships. \emph{R News}, \emph{2}(3), 7--10.
\url{https://CRAN.R-project.org/doc/Rnews/}

\end{CSLReferences}




\end{document}
